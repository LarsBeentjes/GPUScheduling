\documentclass[10pt]{article}

\setlength{\textheight}{25.7cm}
\setlength{\textwidth}{16cm}
\setlength{\unitlength}{1mm}
\setlength{\topskip}{2.5truecm}
\topmargin 260mm \advance \topmargin -\textheight 
\divide \topmargin by 2 \advance \topmargin -1in 
\headheight 0pt \headsep 0pt \leftmargin 210mm \advance
\leftmargin -\textwidth 
\divide \leftmargin by 2 \advance \leftmargin -1in 
\oddsidemargin \leftmargin \evensidemargin \leftmargin
\parindent=0pt

\frenchspacing

\usepackage[english]{babel}

\usepackage{listings}
\lstset{language=C++, showstringspaces=false, basicstyle=\small,
  numbers=left, numberstyle=\tiny, numberfirstline=false, breaklines=true,
  stepnumber=1, tabsize=4, 
  commentstyle=\ttfamily, identifierstyle=\ttfamily,
  stringstyle=\itshape}

\title{GPU server monitoring package documentation}
\author{
Rob Reijtenbach, s1568159 \and 
Elgar R. van der Zande, s1485873 \and
Kean Tettelaar, s1500325 \and 
Luc S. de Jonckheere, s1685538 \and 
Lars J. Beentjes, s1530186 \and 
Gilles B. Ottervanger, s1309773
}

\begin{document}

\selectlanguage{english}

\maketitle

\section{What is it?}

This package has one main goal: Give users insight into jobs running 
on the GPU servers at LIACS. This goal is achieved in two ways: 1) 
It provides a quick and clean view of the running jobs on the GPU's
and 2) It allows teachers to automatically notify users of running
processes according to adjustable rule sets.\\

The package consists of three parts.
\begin{enumerate}
\item{} The GPUMonitor gathers the data of the GPU hardware and the processes 
using them and provides an interface for the other parts of the package 
to access this data quickly. A single instance of the GPUMonitor is 
meant to be running at all times on each GPU server.\\

\item{} The GPUView provides a human readable interface in the shell for a 
selection of the data gathered by the GPUMonitor. Any user on the server
should be able to run the GPUView.\\

\item{} The Violation Detector allows teachers to set notification rules for
processes utilizing GPU hardware. It enables teachers to notify users
of long running jobs, jobs running on multiple GPU's and it allows 
teachers to ask users to keep the servers free for students enrolled
for a specific course.
\end{enumerate}

\section{How to use?}
\subsection{Setup}
The package source files should be placed in a location accessible 
by one of the teachers in order to use the GPUMonitor and the 
Violation Detector. Preferably the GPUView files should be executable
by students from a shared folder.

\subsubsection{GPUMonitor}
The GPUMonitor needs to be run once and keeps running in the background.
The monitor places a socket in the /tmp/ folder providing access to the
gathered data through the MonitorClient. Running the GPUMonitor can be 
done from any account. Trying to run multiple instances of the monitor 
will result in an error. Starting the monitor can be done by running the
/bin/monitor shell script.

\subsubsection{Violation Detector}
The Violation Detector is preferably run by one user, although it is 
possible to run multiple instances. It can be run from the 
/violation\_detector\_env/ folder in order to access all required files.

The Violation detector should run right of the bat, but in order
to configure all the rules properly see `Teacher use case'.
    
\subsection{Student use case}
The student interacts with the system in two separate ways. In the
first place the student can request the GPU utilization information.
This is done by simply calling the /bin/gpuview shell script.
The second form of interaction is simply receiving e-mail notifications
about rule violations and does not require more detailed instructions. 

\subsection{Teacher use case}
The interaction between a Teacher and the system is a little more
complex. In general a Teacher sets RULES applying to GROUPS. When
a user violates a rule a TEMPLATEd e-mail notification is sent to 
this user. \\

RULES, GROUPS and mail TEMPLATES are stored in files. First the syntax of these files is described. Understanding the syntax of these files enables a teacher to do everything the package is intended to do. Finally the syntax is given for the CONFIG file enabling some customization of the default configuration of the violation detector and providing options for testing. 

\subsubsection{RULES}
text file, default location: /violation\_detector\_env/rules.txt

syntax:
\begin{verbatim}
        <rule_type>     <group>     <mail_template>     [parameters]
\end{verbatim}
\verb <rule_type> \\
one of the following:\\
\begin{tabular}{l l}
RESERVE &
- reserve the server for specified group, \\ & all other users
are notified when starting a job.\\
PROC\_TIME & 
- users running jobs longer than the specified time are 
notified. \\ & Time in seconds must be specified in \verb [parameters] .\\
IDLE\_TIME &
- Same as PROC\_TIME but applying to idling time.\\
MAX\_CLAIMED\_GPUS &
- users running jobs claiming more than the specified number 
of devices are notified. \\ & Nr of devices must be specified 
in \verb [parameters] .\\
\end{tabular}

\verb <group> \\
There are roughly two kinds of groups. Default groups and 
custom groups. 

Default groups: (mostly self explanatory)\\
NO\_ONE\\
EVERYONE\\
STUDENTS
- all user names that consist of a lowercase 's' followed 
by any number of digits (eg. s1122334)\\
NOT\_STUDENTS\\

Custom groups:
Custom groups can be defined in a separate file (see GROUPS). Custom 
groups can be used to reserve server for a specific course.
Multiple groups can be separated by commas.\\


\verb <mail_template> \\
Should refer to a .template file in the mailtemplates folder.
The .template extension can be omitted. For template syntax
see TEMPLATES.


Lines in the rules file starting with '\#' are not parsed.
Whitespaces are ignored.
Example:
\begin{verbatim}
        # notify everyone not in PRACTICAL1 when starting jobs with 
        # the message template all_in one.
        RESERVE             PRACTICAL1  all_in_one
        # notify every user running a job for longer than two minutes
        PROC_TIME           EVERYONE    all_in_one                  120
        # notify ELGAR and ROB when idling longer than 20 seconds
        IDLE_TIME           ELGAR,ROB   all_in_one                  20
        # notify every user running a job on 2 or more GPU's
        MAX_CLAIMED_GPUS    EVERYONE    all_in_one                  2
\end{verbatim}
        

\subsubsection{GROUPS}
text file, default location: \verb /violation_detector_env/groups.txt \\
syntax:
\begin{verbatim}
        <groupname>: <users_comma_separated>;
\end{verbatim}
Line breaks and white spaces are ignored, therefore user names can 
be either on separate lines or on the same lines. Groups are
closed by a semicolon.
        
example:
\begin{verbatim}
        PRACTICAL1: s0000000, s0000004, s0000009;
\end{verbatim}
is equivalent to:
\begin{verbatim}
        PRACTICAL1: 
        s0000000,
        s0000004,
        s0000009;
\end{verbatim}
        

\subsubsection{TEMPLATES}
folder containing \verb .template  files,
default location: \verb /violation_detector_env/mailtemplates \\

\verb .template  files are files in the following format:
\begin{verbatim}
        From: [FROM_ADDR]
        To: [TO_ADDR]
        Subject: subject
        
        body
\end{verbatim}
        
Keywords between square brackets are replaced by the mailer
in order to fit the circumstance. The following keywords can be used: \\
\begin{tabular}{ l l }
\lbrack FROM\_ADDR\rbrack      &- sender \\ 
\lbrack TO\_ADDR\rbrack        &- receiver/violator \\
\lbrack FULLNAME\rbrack       &- full name of violator \\
\lbrack SERVER\rbrack         &- server name \\
\lbrack TIME\rbrack           &- Current time (moment of sending e-mail)

\end{tabular}

Example:
\begin{verbatim}
        From: [FROM_ADDR]
        To: [TO_ADDR]
        Subject: Misuse [SERVER]

        Dear [FULLNAME],

        At [TIME] you violated a rule by running a job on [SERVER]. 
        This mail was ment for: [TO_ADDR]

        Best,
        TA
        
        This an automatically generated e-mail.
        For comments or concerns please contact [FROM_ADDR].
\end{verbatim}

\subsubsection{CONFIG file}
\verb .json  file, location: \verb /violation_detector_env/config.json \\
This file enables the user to change the default settings of the violation detector. These settings include file names and debug settings. Below are the possible fields listed with a short description.\\

\begin{tabular}{ l l l }
Name & Default & Description \\ \hline
\verb'groupfile'            &\verb'groups.txt',& group file (see GROUPS)\\
\verb'logfile'              &\verb'mailer.log',& mailer log output file\\
\verb'dup_log_to_stdout'    &\verb'True',& duplicate log output to the terminal  (for testing)\\
\verb'template_dir'         &\verb'mailtemplates',& mail templates directory (see TEMPLATES)\\
\verb'rulefile'             &\verb'rules.txt',& rule file (see GROUPS)\\
\verb'mail_cooldown_file'   &\verb'mailcooldown.json',& temporary file to store recently sent mails\\
\verb'mail_cooldown_time'   &\verb'24 * 3600',& minimal time between mail notifications\\
\verb'mail_dry_run'         &\verb'True' ,& turn off e-mail sending (for testing)\\
\verb'mail_from_addr'       &\verb'unkown@liacs.leidenuniv.nl'& string to replace \verb [FROM_ADDR]  in mail templates\\
\end{tabular}

\end{document}