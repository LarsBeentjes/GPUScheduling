\documentclass[10pt]{article}

\setlength{\textheight}{25.7cm}
\setlength{\textwidth}{16cm}
\setlength{\unitlength}{1mm}
\setlength{\topskip}{2.5truecm}
\topmargin 260mm \advance \topmargin -\textheight 
\divide \topmargin by 2 \advance \topmargin -1in 
\headheight 0pt \headsep 0pt \leftmargin 210mm \advance
\leftmargin -\textwidth 
\divide \leftmargin by 2 \advance \leftmargin -1in 
\oddsidemargin \leftmargin \evensidemargin \leftmargin
\parindent=0pt

\frenchspacing

\usepackage[english]{babel}

\usepackage{listings}
\lstset{language=C++, showstringspaces=false, basicstyle=\small,
  numbers=left, numberstyle=\tiny, numberfirstline=false, breaklines=true,
  stepnumber=1, tabsize=4, 
  commentstyle=\ttfamily, identifierstyle=\ttfamily,
  stringstyle=\itshape}

\title{GPU server monitoring package documentation}
\author{SE Team 10}

\begin{document}

\selectlanguage{english}

\maketitle

\section{What is it?}

    This package has one main goal: Give users insight into jobs running 
    on the GPU servers at LIACS. This goal is achieved in two ways: 1) 
    It provides a quick and clean view of the running jobs on the GPU's
    and 2) It allows teachers to automatically notify users of running
    processes according to ajustable rule sets.\\

    The package consists of three separate parts.\\
    The GPUMonitor gathers the data of the GPU hardware and the processes 
    using them and provides an interface for the other parts of the package 
    to access this data quickly. A single instance of the GPUMonitor is 
    meant to be running at all times on each GPU server.\\

    The GPUView provides a human readable interface in the shell for a 
    slection of the data gathered by the GPUMonitor. Any user on the server
    should be able to run the GPUView.\\
    
    The Violation Detector allows teachers to set notification rules for
    processes utilizing GPU hardware. It enables teachers to notify users
    of long running jobs, jobs running on multiple GPU's and it allows 
    teachers to ask users to keep the servers free for students enrolled
    for a specific course.

\section{How to use?}
\subsection{Setup}
The package source files should be placed in a location accessable 
by one of the teachers in order to use the GPUMonitor and the 
Violation Detector. Preferably the GPUView files should be executable
by students from a shared folder.

\subsubsection{GPUMonitor}
The GPUMonitor needs to be run once and keeps running in the background.
The monitor places a socket in the /tmp/ folder providing access to the
gathered data through the MonitorClient. Running the GPUMonitor can be 
done from any account. Trying to run multiple instances of the monitor 
will result in an error. Starting the monitor can be done by running the
/bin/monitor shell script.

\subsubsection{Violation Detector}
The Violation Detector is preferably run by one user allthough it is 
possible to run multiple instances. It can be run from the 
/violation\_detector\_env/ folder in order to access all required files.

The Violation detector should run right of the bat, but in order
to configure all the rules properly see 'Teacher use case'.
    
\subsection{Student use case}
        The student interacts with the system in two separate ways. In the
        first place the student can request the GPU utilization information.
        This is done by simply calling the /bin/gpuview shell script.
        The second form of interaction is simply recieving e-mail notifications
        about rule violations and does not require more detailed instructions. 

\subsection{Teacher use case}
The interaction between a Teacher and the system is a little more
complex. In general a Teacher sets RULES applying to GROUPS. When
a user violates a rule a TEMPLATEd e-mail notification is sent to 
this user. 

RULES, GROUPS and mail TEMPLATES are stored in files. Understanding
the syntax of these files enables a teacher to do everything the
package is intendet to do.

RULES
text file, default location: /violation\_detector\_env/rules.txt

syntax:
\begin{verbatim}
        <rule_type>     <group>     <mail_template>     [parameters]
\end{verbatim}
\begin{verbatim}<rule_type>\end{verbatim}one of the following:\\
RESERVE\\
- reserve the server for specified group, all other users
are notified when starting a job.\\
PROC\_TIME\\
- users running jobs longer than the specified time are 
notified. \\Time in seconds must be specified in [parameters].\\
IDLE\_TIME\\
- Same as PROC\_TIME but applying to idling time.\\
MAX\_CLAIMED\_GPUS\\
- users running jobs claiming more than the specified number 
of devices are notified. Nr of devices must be specified 
in [parameters].

\begin{verbatim}<group>\end{verbatim}
There are roughly two kinds of groups. Default groups and 
custom groups. 

Default groups: (mosty self explanatory)\\
NO\_ONE\\
EVERYONE\\
STUDENTS
- all usernames that consist of a lowercase 's' followed 
by any number of digits (eg. s1122334)\\
NOT\_STUDENTS\\

Custom groups:
Custom groups can be defined in a separate file (see GROUPS). Custom 
groups can be used to reserve server for a specific course.

Multiple groups can be separated by commas.


\begin{verbatim}<mail_template>\end{verbatim}
Should refer to a .template file in the mailtemplates folder.
The .template extention can be omitted. For template syntax
see TEMPLATES.


Lines in the rules file starting with '\#' are not parsed.
Whitespaces are ignored.
Example:
\begin{verbatim}
        # notify everyone not in PRACTICAL1 when starting jobs with 
        # the message template all_in one.
        RESERVE             PRACTICAL1  all_in_one
        # notify every user running a job for longer than two minutes
        PROC_TIME           EVERYONE    all_in_one                  120
        # notify ELGAR and ROB when idling longer than 20 seconds
        IDLE_TIME           ELGAR,ROB   all_in_one                  20
        # notify every user running a job on 2 or more GPU's
        MAX_CLAIMED_GPUS    EVERYONE    all_in_one                  2
\end{verbatim}
        

GROUPS
text file, default location: /violation\_detector\_env/groups.txt
        
syntax:
\begin{verbatim}
        <groupname>: <users_comma_separated>;
\end{verbatim}
Linebreaks and whitespaces are ignored, therefore usersnames can 
be either on separate lines or on the same lines. Groups are
closed by a semicolon.
        
example:
\begin{verbatim}
        PRACTICAL1: s0000000, s0000004, s0000009;
\end{verbatim}
is equivalent to:
\begin{verbatim}
        PRACTICAL1: 
        s0000000,
        s0000004,
        s0000009;
\end{verbatim}
        

TEMPLATES
folder containing .template files,
default location: /violation\_detector\_env/mailtemplates
        
.template files are files in the following format:
\begin{verbatim}
        From: [FROM_ADDR]
        To: [TO_ADDR]
        Subject: subject
        
        body
\end{verbatim}
        
keywords between square brackets are replaced by the mailer
in order to fit the circumstance. The following keywords can be used: 
[FROM\_ADDR]     - sender 
[TO\_ADDR]       - reciever/violator 
[FULLNAME]      - fullname of violator 
[SERVER]        - server name 
[TIME]          - Current time (moment of sending e-mail) 

Example:
\begin{verbatim}
        From: [FROM_ADDR]
        To: [TO_ADDR]
        Subject: Misuse [SERVER]

        Dear [FULLNAME],

        At [TIME] you violated a rule by running a job on [SERVER]. 
        This mail was ment for: [TO_ADDR]

        Best,
        TA
        
        This an automatically generated e-mail.
        For comments or concerns please contact [FROM_ADDR].
\end{verbatim}

\end{document}