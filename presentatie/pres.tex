\documentclass{beamer}
\usetheme{Madrid}

\title{Process van het maken van een automatische GPU opzichter}
\author{Lars Beentjes \\Luc de Jonckheere \\Gilles Ottervanger \\Rob Reijtenbach \\Kean Tettelaar \\Elgar van der Zande}
\date{19 Januari 2017}

\begin{document}

\frame{\titlepage}


\begin{frame}
\frametitle{Teamwork}
    \begin{itemize}
        \item Regelmatige meetings
        \begin{itemize}
            \item Deadlines stellen
            \item Gemaakt werk evalueren
            \item Idee\"en delen
        \end{itemize}
        \vspace{+2mm}

        \item Whatsapp groep voor snel contact
        \begin{itemize}
            \item Vragen over elkaars code
            \item Vragen over design keuzes
            \item Plannen van meetings
        \end{itemize}
        \vspace{+2mm}

        \item GitHub voor makkelijke samenwerking
        \begin{itemize}
            \item Overzichtelijk code uitwisselen
            \item Issues voor het melden van problemen
        \end{itemize}
    \end{itemize}
\end{frame}


\begin{frame}
\frametitle{Contact met opdrachtgever}
    \begin{itemize}
        \item Eerste meeting
        \begin{itemize}
            \item Iedereen aanwezig
            \item Technische requirements bespreken
            \item Interactie vastleggen (input/output)
            \item Deadlines bespreken
        \end{itemize}
        \vspace{+2mm}

        \item Latere meetings
        \begin{itemize}
            \item Kleinere groep mensen
            \item Product in huidige staat laten zien
            \item Bespreken wat anders moet/beter kan
            \item Eerder gemaakte requirements bijstellen
        \end{itemize}
    \end{itemize}
\end{frame}


\begin{frame}
\frametitle{Projectmatig werken}
    \begin{itemize}
        \item Combinatie van waterfall en agile
        % In het begin was waterfall handig, maar later werd agile handiger
        
        \item Waterfall:
        \begin{itemize}
            \item Requirements vanaf het begin duidelijk
            \item Alle feutures werden tegelijk gemaakt (losse modules)
            \item Losse stappen in successie (begin)
            \begin{itemize}
                \item Eerst duidelijk requirements bespreken
                \item Met z'n alleen over een goed design gebrainstormd
                \item Design implementeren
                \item Afwijking...
            \end{itemize}
        \end{itemize}
        \vspace{+1mm}

        \item Agile:
        \begin{itemize}
            \item Losse teams (pair programming)
            \item Elke interview werkende software
            \item Geen groot documentatie document, maar duidelijke code met benodigde comments
            \item Nieuw interview na grote verandering
            \item Handige/efficiente methode door grootte van het project  % kleine projecten zijn erg wendbaar
        \end{itemize}
    \end{itemize}
\end{frame}


\begin{frame}
\frametitle{Requirement engineering}
    \begin{itemize}
        \item Requirements eenduidig; technieken niet erg nodig
        \vspace{+2mm}

        \item Functionalistische benadering  % objective + order; The analyst is the expert who empirically seeks the truth
        \vspace{+2mm}

        \item \textbf{Elicitatie}: interview met opdrachtgever en brainstorming
        \item \textbf{Specificatie}: requirements uitgewerkt en opgeschreven
        \item \textbf{Validatie}: genoteerde requirements naar opdrachtgever gemaild
        \item \textbf{Onderhandelingen}: e-mails met opdrachtgever en interviews
        \vspace{+2mm}

        \item Weinig sprake van MoSCoW  % we hadden 3 belangrijke modules en wat kleine dingetjes die nog zouden kunnen als we klaar waren
    \end{itemize}
\end{frame}


\begin{frame}
\frametitle{Design}
    \begin{itemize}
        \item Model-View-Controller achtig design patroon  % Niet helemaal, want 3 losse programmas verbonden met socket. Geen directe referencies naar elkaar
        \begin{itemize}
            \item Model: GPUMonitor  % data handling en storage
            \item View: GPUViewer  % Presentatie van de assets
            \item Controller: Violation Detector  % Ontvangt input van model en update info voor viewer
        \end{itemize}
        \vspace{+2mm}

        \item Tegelijk ontwikkelen goed mogelijk door design keuze
        \item Onderdeel kan makkelijk vervangen worden als het niet voldoet
        \vspace{+2mm}

        \item Geen modeling nodig gehad door kleine codebase en duidelijke code(structuren)
        \vspace{+2mm}

        \item Code achteraf refactored om ``bad smells'' te minimaliseren
    \end{itemize}
\end{frame}


\begin{frame}
\frametitle{Kwaliteit}
    \begin{itemize}
        \item Opdrachtgever/ontwikkelings manager:
        \begin{itemize}
            \item Gevraagd probleem opgelost
            \item Code makkelijk aan te passen/goed te onderhouden  % costing less to develop and maintain
        \end{itemize}
        \vspace{+1mm}

        \item Gebruiker:
        \begin{itemize}
            \item Makkelijk te gebruiken  % easy to learn
            \item Duidelijke handleiding meegegeven  % easy to learn
            \item Snel automatiseren van normaal handmatige taken  % helps get work done; efficient to use
        \end{itemize}
        \vspace{+1mm}

        \item Ontwikkelaar:
        \begin{itemize}
            \item Makkelijk design  % easy to design
            \item Logisch design  % easy to maintain
            \item Losse onderdelen, dus makkelijk om delen te hergebruiken  % easy to reuse its parts
        \end{itemize}
    \end{itemize}
\end{frame}


\begin{frame}
\frametitle{Testen}
    \begin{itemize}
        \item Verificatie:
        \begin{itemize}
            \item Lastig te automatiseren in dit project
            \item Weinig mogelijkheden voor input (per module); makkelijk handmatig te testen
            \item Modulaire opbouw zorgt voor consistentie in werking
        \end{itemize}
        \vspace{+2mm}

        \item Validatie:
        \begin{itemize}
            \item Idle script: kan een process op bepaalde GPUs idle laten draaien
            \item Stress test: start process die veel werk uitvoeren
        \end{itemize}
    \end{itemize}
\end{frame}

\end{document}
